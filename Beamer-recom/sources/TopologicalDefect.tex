\documentclass{standalone}
\usepackage{tikz}



\begin{document}

\newcommand{\defect}[2]{
	\pgfmathsetmacro{\k}{#1}
	\pgfmathsetmacro{\c}{#2}
	\draw (0,-8.3)node{$(#1,#2)$}; %ラベル
	
	\filldraw (0,0) circle(0.3);
	
	\foreach \x in {-6,-4,-2,0,2,4,6}
	\foreach \y in {-6,-4,-2,0,2,4,6}{
		\pgfmathparse{not(equal(\x,0))}
		\ifnum\pgfmathresult=1
			\pgfmathparse{\x>0}
			\ifnum\pgfmathresult=1
				\pgfmathsetmacro{\p}{atan(\y/\x)}
			\else
				\pgfmathsetmacro{\p}{atan(\y/\x)+180}
			\fi
			\pgfmathsetmacro{\t}{\k*\p+\c}
			\draw[very thick] (\x,\y) +(\t:-0.9) -- +(\t:0.9);
		\else
			\pgfmathparse{not(equal(\y,0))}
			\ifnum\pgfmathresult=1
				\pgfmathparse{\y>0}
				\ifnum\pgfmathresult=1
					\pgfmathsetmacro{\p}{90}
				\else
					\pgfmathsetmacro{\p}{270}
        \fi
        \pgfmathsetmacro{\t}{\k*\p+\c}
        \draw[very thick] (\x,\y) +(\t:-0.9) -- +(\t:0.9);
		  \fi
	  \fi
  }
}


\begin{tikzpicture}[scale=0.25]
  \pgfmathsetmacro{\xshift}{450} %どれだけx方向にずらすか
  \pgfmathsetmacro{\yshift}{-500} %どれだけy方向にずらすか
  \pgfmathsetmacro{\X}{0}
  \pgfmathsetmacro{\Y}{0}
  
  \foreach \k/\c[remember=\X, remember=\Y] in %k,cを動かす
  {1/0, 1/30, 1/60, 1/90,
  -1/0, -1/30, -1/60, -1/90,
  {1/2}/0, {-1/2}/0, 2/0, -2/0,
  {3/2}/0, {-3/2}/0, 3/0, -3/0}{
    \begin{scope}[xshift=\xshift*\X, yshift=\yshift*\Y]
      \defect{\k}{\c} %描画
    \end{scope}
    
    \pgfmathparse{not(equal(\X,3))}
    \ifnum \pgfmathresult=1 %X!=3
      \pgfmathsetmacro{\X}{\X+1} %Xを1増やす
    \else %X=3
      \pgfmathsetmacro{\X}{0} %X=0にリセット
      \pgfmathsetmacro{\Y}{\Y+1} %Yを1増やす
    \fi
  }
\end{tikzpicture}


\end{document}
