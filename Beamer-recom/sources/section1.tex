\documentclass[./main]{subfiles}

% \graphicspath{{\subfix{../figures/section1/}}}
% \setbeameroption{show notes on second screen=right}

\begin{document}

\setcounter{section}{0}
\section{Beamer}
\subsection{When}
\begin{frame}
  \frametitle{\subsecname\ \secname ?}

  \centering
  スライドが作りたいとき

  \vspace{1ex}

  When you want to make slides

\end{frame}


\subsection{Where}
\begin{frame}
  \frametitle{\subsecname\ \secname ?}

  \centering

  \LaTeX があればどこでも

  \vspace{1ex}

  Anywhere with \LaTeX

\end{frame}


\subsection{Who}
\begin{frame}
  \frametitle{\subsecname\ \secname ?}

  \centering

  誰でも

  \vspace{1ex}

  Anyone who wants

\end{frame}


\subsection{What}
\begin{frame}[fragile]
  \frametitle{\subsecname\ \secname ?}

  \centering

  ドイツ語で「投影機」

  \vspace{1ex}

  \LaTeX に基づき,プレゼンテーションを\\
  作成するためのクラスである\citep{WikiBeamerJa}

  \vspace{1ex}

  A LaTeX document class \\
  for creating presentation slides \citep{WikiBeamerEn}


\end{frame}


\subsection{Why}
\begin{frame}[fragile]
  \frametitle{\subsecname\ \secname ?}

  \begin{itemize}
    \item \LaTeX コマンドが使える
    \begin{itemize}
      \item 数式がきれい
      \item 全体の構造/階層の理解
    \end{itemize}
    \item スライドの統一性の向上
    \begin{itemize}
      \item 実用的な \verb|theme| が豊富に用意されている
      \item 全体のレイアウト・色・フォントも一括で変更可
      \item OverlayとDynamic効果
    \end{itemize}
    \item 講義ノート$\leftrightarrow$プレゼン資料の変換可
    \item pdf出力$\Rightarrow$表示のデバイスやバージョン依存なし
  \end{itemize}

\end{frame}


\subsection{How}
\begin{slide}
  \frametitle{\subsecname\ \secname ?}

  \vfilll

  \begin{itemize}
    \item Lua\LaTeX \footnotemark
  \end{itemize}

  \centering
  \begin{minipage}[t]{0.4\linewidth}
    \small{
      \begin{verbatim}
\documentclass{beamer}

\begin{document}
    \begin{frame}
        Hello, \LaTeX!
    \end{frame}
\end{document}
      \end{verbatim}
    }
  \end{minipage}
  \hfill
  \begin{minipage}[t]{0.55\linewidth}
    \small{
      \begin{verbatim}
\documentclass[unicode]{beamer}
\usepackage{luatexja}
\begin{document}
    \begin{frame}
        こんにちは,\LaTeX!
    \end{frame}
\end{document}
      \end{verbatim}
    }
  \end{minipage}
  
  \vfilll
  \vfilll

  {\scriptsize
    \hfill\footnotemark[1]{Lua\LaTeX って何?な方は別紙「てふてふ」参照.}
  }

  \framebreak

  \begin{itemize}
    \item どうしてもup\LaTeX
  \end{itemize}

  \centering
  \begin{verbatim}
      \documentclass[dvipdfmx]{beamer}
      \usepackage{bxdpx-beamer}
      \begin{document}
          \begin{frame}
              こんにちは,\LaTeX!
          \end{frame}
      \end{document}
  \end{verbatim}

\end{slide}





\begin{frame}[fragile]
  \frametitle{\subsecname\ \secname $\bm{+\alpha}$}

  \centering

  \begin{minipage}{0.3\linewidth}
    {\small 
      \begin{block}<1->{block}
        ABCDEFG
      \end{block}
      \begin{alertblock}<2>{alert block}
        いろはにほへと
      \end{alertblock}
      \begin{exampleblock}<3->{example block}
        あいうえお
      \end{exampleblock}
    }
  \end{minipage}
  \hfill
  \begin{minipage}{0.6\linewidth}
    \begin{beamercolorbox}[wd=\linewidth, rounded=true]{grey box}
      {\small
        \begin{verbatim}
\begin{block}<1->{block}
  ABCDEFG
\end{block}
\begin{alertblock}<2>{alert}
  いろはにほへと
\end{alertblock}
\begin{exampleblock}<3->{example}
  あいうえお
\end{exampleblock}
        \end{verbatim}%
      }%
      \vspace{-15pt}
    \end{beamercolorbox}
  \end{minipage}

\end{frame}


\begin{frame}[fragile]
  \frametitle{\subsecname\ \secname $\bm{+\alpha}$}

  Overlay: \uncover<2->{uncover} \only<3->{only} \visible<4->{visible} \invisible<5->{invisible} \alt<6->{\alert{alt}}{alt} \temporal<3-4>{before}{temporal}{after}

  {\small
    \begin{verbatim}
\uncover<2->{uncover} \only<3->{only} 
\visible<4->{visible} \invisible<5->{invisible} 
\alt<6->{\alert{alt}}{alt} 
\temporal<3-4>{before}{temporal}{after}
    \end{verbatim}
  }

  \begin{table}[htbp]
    \pause
    \centering\begin{tabular}{ll}\bhline{1pt}
      \verb|\uncover| $=$ \verb|\onslide| & 半透明(Dynamic効果) \pause\\
      \verb|\only| $=$ \verb|\onslide*| & 存在しなくなる \pause\\
      \verb|\visible| $=$ \verb|\onslide+| & 空白になる \\ \bhline{1pt}
    \end{tabular}
  \end{table}

\end{frame}



\begin{frame}[fragile]
  \frametitle{\subsecname\ \secname $\bm{+\alpha}$}


  \centering

  \begin{minipage}{0.35\linewidth}
    \begin{enumerate}[<+-|alert@+>]
      \item foo
      \item bar
      \begin{enumerate}
        \item baz
        \item qux
          \begin{enumerate}
          \item quxx
          \end{enumerate}
        \end{enumerate}
      \item corge
    \end{enumerate}  
  \end{minipage}
  \begin{minipage}{0.6\linewidth}
    \begin{beamercolorbox}[wd=\linewidth, rounded=true]{grey box}
      {\small
        \begin{verbatim}
\begin{enumerate}[<+-|alert@+>]
  \item foo
  \item bar
  \begin{enumerate}
    \item baz
    \item qux
      \begin{enumerate}
      \item quxx
      \end{enumerate}
    \end{enumerate}
  \item corge
\end{enumerate}  
        \end{verbatim}
      }%
      \vspace{-15pt}
    \end{beamercolorbox}%
  \end{minipage}
  
\end{frame}



\ifSubfilesClassLoaded{%
  \begin{frame}
    \bibliography{../ref}
  \end{frame}
}{}


\end{document}