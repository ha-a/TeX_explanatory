% +++
% sequence = ["latex", "bibtex", "latex", "latex"]
% [programs.latex]
% 	command = "lualatex"
% 	opts = ["-synctex=1", "-file-line-error", "-interaction=nonstopmode"]
% 	args = ["%S"]
% [programs.bibtex]
% 	command = "upbibtex"
% 	target = "../ref.bib"
% 	args = ["%B"]
% +++

\documentclass[./main]{subfiles}
\graphicspath{{\subfix{./figures/section5/}}}
\setcounter{section}{4}

\begin{document}
\section{ちょっとしたTips}
\addtocontents{lof}{\protect\addvspace{1em}}
\noindent
ここまでで,基本的な\TeX の実行について幾分イメージがつきやすくなったと思う.
これより先は「へえ、そうだったのか」と個人的に思ったような\TeX 周りの何事か(学会原稿周辺のルールなども含む)を完全に備忘録として記していく.

\subsection{\LuaLaTeX でなんとやら}
\subsubsection{フォント}
本資料でもフォントの調整は行っている.
このように途中で変えるのもお手の物である.


\ifSubfilesClassLoaded{%
  \printbibliography
}% 

\end{document}
