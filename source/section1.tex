% +++
% sequence = ["latex", "biber", "latex"]
% [programs.latex]
% 	command = "lualatex"
% 	opts = ["-synctex=1", "-file-line-error", "-interaction=nonstopmode"]
% 	args = ["%S"]
% [programs.biber]
% 	command = "biber"
% 	target = "../ref.bib"
% 	args = ["--bblencoding=utf8", "-u", "-U", "--output_safechars", "%B"]
% +++

\documentclass[./main]{subfiles}
\graphicspath{{\subfix{./figures/section1/}}}
\setcounter{section}{0}
\begin{document}

\section{はじめに}
\noindent
本資料は\TeX や \LaTeX ,\verb|latexmk|の中身やコンパイルとはなにかについて,備忘録の意味も込めて概説するものです.
インストールの方法などについて解説するわけではなく,これを読んだら環境構築が楽になるわけでもなければ,\verb|tex|が上手に書けるようになるわけでもないです.
これまで気にせずに使っていたけど,よく考えると何も知らなくて,たまに理不尽に言うことを聞いてくれないから怖い``アイツ''についてほんの少しでもその中身と歴史を知ろうというわけです.

実際ネット上には,\TeX と\LaTeX の区別がついてないものや,使ってはいけないもしくは強く推奨されないコマンドを紹介しているようなブログやQiita記事がたくさんあります.
というか,当時正しくても今となっては正しくないものも多かったりします.
そんなことを念頭に本資料ではなるべく,少なくとも執筆当初は嘘ではないこと,を書くようにしているつもりです.
が,私もガチプロではないので,間違えてることもあるかもしれません.
もし間違いを見つけた場合はお近くの私までご連絡をお願いします.


\ifSubfilesClassLoaded{%
  \printbibliography
}{}


\end{document}
