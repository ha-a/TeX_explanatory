% +++
% sequence = ["latex", "biber", "latex"]
% [programs.latex]
% 	command = "lualatex"
% 	opts = ["-synctex=1", "-file-line-error", "-interaction=nonstopmode"]
% 	args = ["%S"]
% [programs.biber]
% 	command = "biber"
% 	target = "../ref.bib"
% 	args = ["--bblencoding=utf8", "-u", "-U", "--output_safechars", "%B"]
% +++

\documentclass[./main]{subfiles}
\graphicspath{{\subfix{./figures/section1/}}}
\setcounter{section}{0}
\begin{document}

\section{はじめに}
\noindent
本資料は\TeX や \LaTeX ,\verb|latexmk|の中身やコンパイルとはなにかについて,備忘録の意味も込めて記録するものである.
インストールの方法などについて解説するものではなく,これを読んだら環境構築で間違わないわけでもなければ,\verb|tex|が上手に書けるようになるわけでもない.
普段気にせずに使っていたけど,よく考えると何も知らなくて,たまに理不尽に言うことを聞いてくれないから怖い``アイツ''について「頭をなでたことがあります」と他人に言っても差し支えないくらいの知識をつけることが目的である\footnote{あえて言うならば「抱きかかえる」ことや「素っ裸にする」こともできてはいないことには留意しなくてはならない.}.

\TeX や\LaTeX の区別や使ってはいけない,もしくは強く推奨されないコマンドを紹介しているようなブログやQiita記事はたくさん溢れている.
当時正しくても今となっては正しくないものも多い.
そんなことを念頭に本資料ではなるべく嘘は書かないようにしているつもりである.
しかし,私も``学生の中では比較的詳しい方''なだけで,``とても詳しい偉い人''ではないので間違えていることもあるかもしれない.
間違いを見つけた場合はお近くの私までご連絡をお願いいたします.
また,次節で説明する通り,\TeX や\LaTeX とその周辺は今なお進化を続けている``生きた言語''であり,最新の情報やトレンドによって変化していくものだが,この資料の内容を常に最新情報にするだけの気概はないため,あくまで資料作成時の情報だということについてもご了承願いたい.


\ifSubfilesClassLoaded{%
  \printbibliography
}{}


\end{document}
