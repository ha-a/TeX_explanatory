% +++
% sequence = ["latex", "bibtex", "latex", "latex"]
% [programs.latex]
% 	command = "lualatex"
% 	opts = ["-synctex=1", "-file-line-error", "-interaction=nonstopmode"]
% 	args = ["%S"]
% [programs.bibtex]
% 	command = "upbibtex"
% 	target = "../ref.bib"
% 	args = ["%B"]
% +++

\documentclass[./main]{subfiles}

\begin{document}
\appendix
\addtocontents{toc}{\protect\setcounter{tocdepth}{2}}
\renewcommand{\thesubsection}{\Alph{subsection}}
\makeatletter
	\renewcommand{\theequation}{\thesubsection.\arabic{equation}}
	\@addtoreset{equation}{subsection}
	\renewcommand{\thefigure}{\thesubsection.\arabic{figure}}
	\@addtoreset{figure}{subsection}
	\renewcommand{\thetable}{\thesubsection.\arabic{table}}
	\@addtoreset{table}{subsection}
\makeatother

\setcounter{equation}{0}

%\renewcommand{\thesubsection}{\Alph{subsection}}
\section*{\appendixname}
\addcontentsline{toc}{section}{\appendixname}
\addtocontents{lof}{\protect\addvspace{1em}}
\addtocontents{lot}{\protect\addvspace{1em}}

\markboth{\appendixname}{}

\subsection{はじめてのらてふかんきょうづくり}

\begin{itemize}
  \item Windows(要検証)
  \begin{enumerate}
    \item TeX Liveのページ(\url{https://www.tug.org/texlive/acquire-netinstall.html})から\verb|install-tl-windows.exe|をダウンロード.
    \item ダウンロードしたファイルを実行.のち,待ち.
  \end{enumerate}
  \item Mac
  \begin{enumerate}
    \item Homebrewをダウンロード.\\
    \vspace{-10pt}
    \begin{verbatim}
/bin/bash -c "$(curl -fsSL 
          https://raw.githubusercontent.com/Homebrew/install/HEAD/install.sh)"
    \end{verbatim}
    \vspace{-10pt}
    \item MacTeXをインストール.のち,待ち.\\
    \vspace{-10pt}
    \begin{verbatim}
brew install --cask mactex-no-gui
sudo tlmgr update --self --all
sudo tlmgr paper a4
    \end{verbatim}
    \vspace{-10pt}
  \end{enumerate}
\end{itemize}
\begin{enumerate}
  \setcounter{enumi}{2}
  \item ここで一旦確認.例えば\quad
  \begin{minipage}[t]{0.2\linewidth}
    \verb|main.tex|
    \begin{lstlisting}[numbers=none,linewidth=\linewidth]
\documentclass{article}
\begin{document}
	Hello, \LaTeX!
\end{document}      
    \end{lstlisting}
  \end{minipage}
  を作って,コンパイル:\quad
  \begin{minipage}[t]{0.2\linewidth}
    \begin{verbatim}
latex main.tex
dvipdfmx main.dvi
open main.pdf
    \end{verbatim}
  \end{minipage}
  \item \LuaLaTeX も確認.
  \verb|lualatex|\quad
  \begin{minipage}[t]{0.45\linewidth}
    \verb|main_lua.tex|\quad
    \begin{lstlisting}[numbers=none,linewidth=\linewidth]
\documentclass[a4paper,lualatex,ja=standard]{bxjsarticle}
\begin{document}
  Hello, \LaTeX! こんにちは,\LaTeX!
\end{document}      
    \end{lstlisting}
  \end{minipage}
  \item \verb|latexmk|を使うための設定ファイル\verb|latexmkrc|を作成.\LuaLaTeX なら以下.\\
  \begin{figure}[h]
  \centering
  \begin{minipage}{0.65\linewidth}
  \begin{lstlisting}[commentstyle={\color[rgb]{0,0,0}},]
#!/usr/bin/env perl
$latex     = 'uplatex  %O -synctex=1 -file-line-error -halt-on-error %S';
$pdflatex  = 'pdflatex %O -synctex=1 -file-line-error -halt-on-error %S';
$lualatex  = 'lualatex %O -synctex=1 -file-line-error -halt-on-error %S';
$xelatex   = 'xelatex  %O -synctex=1 -file-line-error -halt-on-error %S';
$biber     = 'biber    %O --bblencoding=utf8 -u -U --output_safechars %B';
$bibtex    = 'upbibtex %O %B';
$dvipdf    = 'dvipdfmx %O -o %D %S';
$pdf_mode  = 4; # or 3 (for upLaTeX)
  \end{lstlisting}
\end{minipage}
\end{figure}
  \upLaTeX なら9行目の\verb|pdf_mode|を$3$にする.
  \item もっかい確認.\verb|latexmk main.tex|でコンパイル.\verb|main.tex|には相互参照や文献参照を追加してもいい.
  \LuaLaTeX なら\verb|latexmk -lualatex main_lua.tex|
  \item Visual Studio Codeをインストール.
  \item VSCodeで拡張機能LaTeX Workshopをインストール.
  \item VSCodeでコマンドパレットから\verb|settings.json|を開き,以下を追加する:
  \begin{figure}[h]
    \centering
    \begin{minipage}{0.3\linewidth}
    \begin{lstlisting}[commentstyle={\color[rgb]{0,0,0}},]
"latex-workshop.latex.recipes": [
  {
      "name": "latexmk",
      "tools": [
          "latexmk_tool"
      ]
  },
],
"latex-workshop.latex.tools": [
  {
      "name": "latexmk_tool",
      "command": "latexmk",
      "args": [
          "-outdir=%OUTDIR%",
          "%DOC%"
      ],
  },
],
    \end{lstlisting}
  \end{minipage}
  \end{figure}
  \item 最後の確認.VSCodeで\verb|main.tex|を開き,コマンドパレットから\verb|Build LaTeX project|を実行.
\end{enumerate}

これでとりあえず\upLaTeX でも\LuaLaTeX でもその他のなにかでも動く!はず(\verb|latexmkrc|は文書に合わせて適宜変える必要あり).

高尾湖から先はすべて個人の好みである.例えば
\begin{itemize}
  \item VSCodeの設定ファイル\verb|settings.json|の\verb|recipes|と\verb|tools|は好きなように組合せて作れるため,
  \begin{itemize}
    \item ログファイルを出力しないようにする
    \item \LuaLaTeX\ $+$ \verb|latexmk|
    \item \upLaTeX\ $+$ \verb|latexmk|
    \item \upLaTeX\ $+$ \upBibTeX\ $+$ \upLaTeX $\times2$ $+$ \verb|dvipdfmx|
  \end{itemize}
  とかを用意しておき,適宜ショートカットで切り替えながら使う.
  \item 出力を別フォルダにすることで,中間ファイルでフォルダがごちゃごちゃするのを抑制する.
  \item 保存するたびに自動でコンパイルさせる.
  \item 出力PDFの表示をVSCodeのプレビューで行う.もしくは外部アプリで行う.
\end{itemize}
などなど色々工夫ができる.





\end{document}
